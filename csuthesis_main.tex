\documentclass{CSUthesis}

\addbibresource{thesis-references.bib}

\include{content/info}

% lipsum
\newcommand{\lipsum} {
    这是一段随机插入的文本,用来填充模板布局,感受模板视觉效果。
    
    测试红楼梦\cite{蔡义江2007红楼梦}

    中南大学坐落在中国历史文化名城──湖南省长沙市,占地面积5886亩,建筑面积276万平方米,跨湘江两岸,依巍巍岳麓,临滔滔湘水,环境幽雅,景色宜人,是求知治学的理想园地。
  中南大学是教育部直属全国重点大学、国家“211工程”首批重点建设高校、国家“985工程”部省重点共建高水平大学和国家“2011计划”首批牵头高校,2017年9月经国务院批准入选世界一流大学A类建设高校。现任校党委书记易红、校长田红旗。
  中南大学由原湖南医科大学、长沙铁道学院与中南工业大学于2000年4月合并组建而成。原中南工业大学的前身为创建于1952年的中南矿冶学院,原长沙铁道学院的前身为创建于1953年的中南土木建筑学院,两校的主体学科最早溯源于1903年创办的湖南高等实业学堂的矿科和路科。原湖南医科大学的前身为1914年创建的湘雅医学专门学校,是我国创办最早的西医高等学校之一。
  中南大学秉承百年办学积淀,顺应中国高等教育体制改革大势,弘扬以“知行合一、经世致用”为核心的大学精神,力行“向善、求真、唯美、有容”的校风,坚持自身办学特色,服务国家和社会重大需求,团结奋进,改革创新,追求卓越,综合实力和整体水平大幅提升。
  

    这是一段随机插入的文本,用来填充模板布局,感受模板视觉效果。
}

\begin{document}
%%%%%%%%%%%%%%%%%%%%%%%%%%%%%%%%%%%%%%%%%%%%%%%%%%
% 封面
% -----------------------------------------------%
\makecoverpage

%%%%%%%%%%%%%%%%%%%%%%%%%%%%%%%%%%%%%%%%%%%%%%%%%%
% 前置部分的页眉页脚设置
% -----------------------------------------------%
\newpage
% 正文和后置部分用阿拉伯数字编连续码,前置部分用罗马数字单独编连续码(封面除外)。
% 设置封面页后的页码
\pagenumbering{Roman} % 大写罗马字母
\setcounter{page}{1} % 从1开始编号页码
% 设置页眉和页脚
% 本科生从摘要开始就要有
% 设置页眉和页脚 %
\pagestyle{fancy}
\fancyhead[L]{\includegraphics[scale=0.15]{cover_icon.png}}
\fancyhf[RH]{\heiti \zihao{5} {图像与激光融合的轨道扣件脱落检测}} % 设置所有(奇数和偶数)右侧页眉

% 封面页无需页码,其他前置部分需要(按此理解扉页也是要页码的)。
% totalPage是一个标签,把它放在正文的最后面
\fancyfoot[C]{\songti 第 \thepage 页\quad 共\pageref{totalPage} 页} % 所有(奇数和偶数)中间页脚


%%%%%%%%%%%%%%%%%%%%%%%%%%%%%%%%%%%%%%%%%%%%%%%%%%
% 中文摘要
% -----------------------------------------------%
\include{content/abstractcn}
\newpage

%%%%%%%%%%%%%%%%%%%%%%%%%%%%%%%%%%%%%%%%%%%%%%%%%%
% 英文摘要
% -----------------------------------------------%
\include{content/abstracten}
\newpage

% 目录
% -------------------------------------------%

{

\renewcommand{\contentsname}{\hfill \heiti \zihao{3} 目\quad 录\hfill}
	\renewcommand*{\baselinestretch}{1.5}   % 行间距
    \tableofcontents
}
\newpage
% 去掉页眉章节序号后面的“.”
\renewcommand{\sectionmark}[1]{\markright{\thesection~ #1}}


\renewcommand{\headrulewidth}{1pt}

% 正文内容
% --------------------------------------------%
\setcounter{page}{1} % 重置页码编号
\pagenumbering{arabic} % 设置页码编号为阿拉伯数字
\fancyfoot[C]{\songti 第 \thepage 页\quad 共\pageref{totalPage} 页} % 所有(奇数和偶数)中间页脚

% 可以使用include命令导入tex文件,从而避免过多修改本文件。

% 论文正文是主体,主体部分应从另页右页开始,每一章应另起页。一般由序号标题、文字叙述、图、表格和公式等五个部分构成。

% 重新设置正文行间距,因为前置部分设置时候行间距被改过
\renewcommand*{\baselinestretch}{1.0}   % 几倍行间距
\setlength{\baselineskip}{20pt}         % 基准行间距

% 正文
%!TEX root = ../csuthesis_main.tex

%子章节为了便于查找和修改,建议通过input拆分文件

%%%%%%%%%%%%%%%%%%%%%%%%%%%%%%%%绪论%%%%%%%%%%%%%%%%
\input{content/subchapters/sec1.tex}
%%%%%%%%%%%%%%%%%%%%%%%%%%%%%%%%绪论%%%%%%%%%%%%%%%%

%%%%%%%%%%%%%%%%%%%%%%%%%%%%%%%%图像插入示例%%%%%%%%%%%%%%%%
\input{content/subchapters/sec2.tex}
%%%%%%%%%%%%%%%%%%%%%%%%%%%%%%%%图像插入示例%%%%%%%%%%%%%%%%


%%%%%%%%%%%%%%%%%%%%%%%%%%%%%%%%表格插入示例%%%%%%%%%%%%%%%%
\input{content/subchapters/sec3.tex}
%%%%%%%%%%%%%%%%%%%%%%%%%%%%%%%%表格插入示例%%%%%%%%%%%%%%%%


%%%%%%%%%%%%%%%%%%%%%%%%%%%%%%%%参考文献插入示例%%%%%%%%%%%%%%%%
\input{content/subchapters/sec4.tex}
%%%%%%%%%%%%%%%%%%%%%%%%%%%%%%%%参考文献插入示例%%%%%%%%%%%%%%%%

%%%%%%%%%%%%%%%%%%%%%%%%%%%%%%%%总结插入示例%%%%%%%%%%%%%%%%
\input{content/subchapters/sec5.tex}
%%%%%%%%%%%%%%%%%%%%%%%%%%%%%%%%总结插入示例%%%%%%%%%%%%%%%%


\newpage


%%%%%%%%%%%%%%%%%%%%%%%%%%%%%%%%%%%%%%%%%%%%%%%%%%
% 临时标签,用于编译时追踪正文末尾
%%%%%%%%%%%%%%%%%%%%%%%%%%%%%%%%%%%%%%%%%%%%%%%%%%

%%%%%%%%%%%%%%%%%%%%%%%%%%%%%%%%%%%%%%%%%%%%%%%%%%
% 后续内容,标题三号黑体居中,章节无编号
% --------------------------------------------%

% https://www.zhihu.com/question/29413517/answer/44358389 %
% 说明如下:
% secnumdepth 这个计数器是 LaTeX 标准文档类用来控制章节编号深度的。在 article 中,这个计数器的值默认是 3,对应的章节命令是 \subsubsection。也就是说,默认情况下,article 将会对 \subsubsection 及其之上的所有章节标题进行编号,也就是 \part, \section, \subsection, \subsubsection。LaTeX 标准文档类中,最大的标题是 \part。它在 book 和 report 类中的层级是「-1」,在 article 类中的层级是「0」。这里,我们在调用 \appendix 的时候将计数器设置为 -2,因此所有的章节命令都不会编号了。不过,一般还是会保留 \part 的编号的。所以在实际使用中,将它设置为 0 就可以了。

% 在修改过程中请注意不要破环命令的完整性

\renewcommand\appendix{\setcounter{secnumdepth}{-2}}
\appendix

% 主文件有代码去掉页眉章节编号的“.”,但这会因为bug导致无编号章节显示一个错误编号,所以这里在无编号章节之前再次重定义sectionmark。
\renewcommand{\sectionmark}[1]{\markright{#1}}

% section 标题从这里往后改为三号黑体居中
\titleformat{\section}{\centering \zihao{3}\heiti}{\thesection}{1em}{}

% \section{参考文献} % bibliography会自动显示参考文献四个字
\addcontentsline{toc}{section}{参考文献} % 由于参考文献不是section,这句把参考文献加入目录
% \nocite{*} % 该命令用于显示全部参考文献,即使文中没引用
% cls文件中已经引入package,这里不需要调用 \bibliographystyle 了。
% \bibliographystyle{gbt7714-2005}
% \bibliography{thesis-references}
\printbibliography
\newpage

\include{content/additional}


\end{document}
